\documentclass[12pt,a4paper]{article}
\usepackage{fontspec}
\defaultfontfeatures{Mapping=tex-text}
\usepackage{xunicode}
\usepackage{xltxtra}
%\setmainfont{???}
\usepackage{polyglossia}
\setdefaultlanguage{english}
\author{Katrina Olsen}
\title{Noisy Channel Experiment Report}
\begin{document}
\maketitle

\tableofcontents

\section{Noisy Channel}
A noisy channel experiment explores how two people can communicate information, despite any misleading of confusing elements that may be present. In speech conversations, this is literal noise. In written communication, the term noise is extended to represent anything in the written text that can distract from the intended information, e.g. typos and ungrammatical structures.
\section{Experiment}
To test to what extent a reader can process information when noise is present, a survey was performed where subjects were asked to rate plausibility of sentences that slightly differed grammatically.
\subsection{Methods}
The data was collected via an online survey in April 2024.
\subsubsection{Participants}
The participants of the study were the students attending the Digital Research Toolkit for Linguistics course in the Summer Semester of 2024 at the University of Stuttgart. The students were invited to invite friends and family to participate as well.
\subsubsection{Materials}
The materials for the study were provided by Ted Gibson (Gibson et al. 2013). Example sentences include:
\begin{itemize}
  \item The cook baked Lucy a cake.
  \item The cook baked Lucy for a cake.
  \item The cook baked a cake for Lucy.
  \item The cook baked a cake Lucy.
\end{itemize}
\subsubsection{Procedure}
The survey began with instructions on the process, along with an example. The participant was to hit the space bar after reading one segment of the sentence at a time. After reading the full sentence, they were asked to rate the sentence on its naturalness, 1 being very unnatural and 5 as very natural.
\subsection{Predictions}
Given the design of the survey, it can be predicted that ungrammatical or implausible sentences would receive a lower naturalness score and require a longer reading time from the participant.
\subsection{Analysis and Results}
As expected, plausible sentences were given a higher mean acceptability (4.23) than implausible sentences (1.88). The reading time however was higher for plausible sentences at the final stage of reading than implausible sentences.
\section{Discussion}
It is possible the participants had a longer reading time at the final stage of plausible sentences because they were double checking they hadn't missed something ungrammatical and that the sentence was indeed plausible.
\section{References}
Gibson, Edward, Leon Bergen, and Steven T Piantadosi (2013).
“Rational integration of noisy evidence and prior semantic
expectations in sentence interpretation”. In: Proceedings of the
National Academy of Sciences 110.20, pp. 8051–8056. DOI:
10.1073/pnas.1216438110.
\end{document}